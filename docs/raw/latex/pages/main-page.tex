% SPDX-License-Identifier: GPL-3.0-or-later OR CC-BY-SA-4.0
\section{Main Page}\label{sec:main-page} %%##pages_mainpage-section==title>>
%%!!pages_mainpage-section<<
Main page lists all the installed apps (or a list of apps supplied by any third-party application) as well as apps that
has existing backups. A single click on any installed app item opens the respective \hyperref[sec:app-details-page]{App
Details page}. Using the \hyperlink{par:main-page-sort}{sort} option from the menu, the app items can be sorted in
various ways and preserved on close. It is also possible to filter items based on your needs using the
\hyperlink{par:main-page-filter}{filter} option in the menu. You can filter using more than one options. You can also
filter apps using app labels or package name using the search button.
%%!!>>

\subsection{Batch Operations}\label{subsec:batch-operations} %%##pages_mainpage-batch-operations==title>>
%%!!pages_mainpage-batch-operations<<
Batch operations or operation on multiple apps are also available within this page (most these operations require root
or \hyperref[sec:adb-over-tcp]{ADB over TCP}). To enable multiple selection mode, click on any app icon or long click
on any app. After that, you can use the regular click to select apps instead of opening the App Details page. These
operations include:
\begin{itemize}
    \item Save APK files to \texttt{AppManager/apks}
    \item \hyperref[sec:backup-restore]{Backup/restore/delete app data}
    \item Clear data from apps
    \item Enable/disable/force-stop/uninstall apps
    \item Prevent background operation
    \item Block ad and tracker components
    \item Export blocking rules
    \item Add the selected apps to a \hyperref[sec:profiles-page]{profile}
\end{itemize}
%%!!>>

\subsection{Colour Codes}\label{subsec:main-colour-codes} %%##pages_mainpage-colour-codes==title>>
%%!!pages_mainpage-colour-codes<<
Here's the list of colours used in this page, and their meaning:
\begin{itemize}
    \item \colorbox{AMLightGreyishOrange}{\textcolor{black}{Light greyish orange (day)}} / \colorbox{AMDarkBlue}{
        \textcolor{white}{dark blue (night)}} -- App selected for batch operation
    \item \colorbox{AMLightRed}{\textcolor{black}{Light red (day)}} / \colorbox{AMVeryDarkRed}{
        \textcolor{white}{dark red (night)}} -- Disabled app
    \item \colorbox{AMYellow}{\textcolor{black}{Yellow Star}} -- App is in debug mode
    \item \textcolor{AMOrange}{Orange \textit{Date}} -- App can read logs (permission granted)
    \item \textcolor{AMOrange}{Orange \textit{UID}} -- User ID is being shared between apps
    \item \textcolor{AMOrange}{Orange \textit{SDK}} -- Uses cleartext (ie. HTTP) traffic
    \item \textcolor{red}{Red \textit{package name}} -- App does not allow clearing data
    \item \textcolor{red}{Red \textit{backup}} -- App not installed but has one or more backups
    \item \textcolor{AMOrange}{Orange \textit{backup}} -- Outdated backup, i.e.\ the version code of the backed up app
    is less than the installed version code of the same app
    \item \textcolor{AMDarkCyan}{Dark cyan \textit{backup}} -- Up to date backup, i.e.\ the version code of the backed
    up app is greater than or equal to the installed version code of the same app
    \item \textcolor{AMDarkCyan}{Dark cyan \textit{package name}} -- Stopped or forced closed app
    \item \textcolor{AMDarkCyan}{Dark cyan \textit{version}} -- Inactive app
    \item \textcolor{magenta}{Magenta} -- Persistent app i.e.\ remains running all the time
\end{itemize}
%%!!>>

\subsection{Application Types}\label{subsec:main-page-application-types} %%##pages_mainpage-application-types==title>>
%%!!pages_mainpage-application-types<<
An app is either \textbf{User App} or \textbf{System App} along with the following codes:
\begin{itemize}
    \item \texttt{X} -- The app supports multiple architectures: 32-bit, 64-bit or arm-v7, arm-v8, etc.
    \item \texttt{0} -- App does not have code with it
    \item \texttt{\textdegree} -- App is in suspended state
    \item \texttt{\#} -- App has requested a large heap
    \item \texttt{?} -- App has requested VM in safe mode
\end{itemize}
%%!!>>

\subsection{Version Info}\label{subsec:main-page-version-info} %%##pages_mainpage-version-info==title>>
%%!!pages_mainpage-version-info<<
Version name is followed by the following prefixes:
\begin{itemize}
    \item \texttt{\_} -- No hardware acceleration
    \item \texttt{\textasciitilde} -- Test only mode
    \item \texttt{debug} -- Debuggable app
\end{itemize}
%%!!>>

\subsection{Options Menu}\label{subsec:main-page-options-menu} %%##pages_mainpage-options-menu==title>>
%%!!pages_mainpage-options-menu<<
Options menu has several options which can be used to sort, filter the listed apps as well as navigate to different pages.
%%!!>>

\subsubsection{Instructions} %%##pages_mainpage-instructions==title>>
%%!!pages_mainpage-instructions<<
Clicking on instructions open an offline version of the App Manager documentations. It opens the online version if the
corresponding feature split i.e.\ \texttt{feat\_docs} is not installed.
%%!!>>

\subsubsection{List Options}\label{subsubsec:main-list-options} %%##pages_mainpage-list-options==title>>
%%!!pages_mainpage-list-options<<
List options contain the options to sort or filter the app list.
%%!!>>

\paragraph{Sort}\hypertarget{par:main-page-sort}{} %%##pages_mainpage-sort==title>>
%%!!pages_mainpage-sort<<
Apps listed on the main page can be sorted in different ways. The sorting preference is preserved which means the apps
will be sorted the same way that was sorted in the previous launch. Regardless of your sorting preference, however, the
apps are first sorted alphabetically to prevent random results.
\begin{itemize}
    \item \textbf{User app first.} User apps are listed on top
    \item \textbf{App label.} Sort in ascending order based on the app labels (also known as \textit{app names}). This
    is the default sorting preference
    \item \textbf{Package name.} Sort in ascending order based on package names
    \item \textbf{Last update.} Sort in descending order based on the package update date (or install date if it is a
    newly installed package)
    \item \textbf{Shared user ID.} Sort in descending order based on the kernel user ID
    \item \textbf{Target SDK.} Sort in ascending order based on the target SDK
    \item \textbf{Signature.} Sort in ascending order based on app's signing info
    \item \textbf{Disabled first.} List disabled apps first
    \item \textbf{Blocked first.} Sort in descending order based on the number of blocked components
    \item \textbf{Backup first.} Display apps with backups on top
\end{itemize}

There is also a \textit{reverse} option which can be used to sort the apps in reverse order.
%%!!>>

\paragraph{Filter}\hypertarget{par:main-page-filter}{} %%##pages_mainpage-filter==title>>
%%!!pages_mainpage-filter<<
Apps listed on the main page can be filtered in a number of ways. Like sorting, filtering preferences are also stored
and retained after a relaunch.
\begin{itemize}
    \item \textbf{User apps.} List only the user apps
    \item \textbf{System apps.} List only the system apps
    \item \textbf{Disabled apps.} List only the disabled apps
    \item \textbf{Apps with rules.} List the apps with one or more blocking rules
    \item \textbf{Apps with activities.} List the apps with one or more activities
    \item \textbf{Apps with backups.} List apps with one or more backups (including uninstalled apps)
    \item \textbf{Apps without backups.} List apps without any backups (including uninstalled apps)
    \item \textbf{Running apps.} List apps that are currently running (either in foreground or background)
    \item \textbf{Apps with splits.} List apps with one or more split APKs
    \item \textbf{Installed apps.} List only installed apps
    \item \textbf{Uninstalled apps.} List only uninstalled apps including backups
\end{itemize}

Unlike sorting, you can filter using more than one option. For example, you can list the disabled user apps by filtering
app lists selecting \textit{user apps} and \textit{disabled apps}. This is particularly useful for
\hyperref[subsec:batch-operations]{batch operations} where you can filter all user apps to carry out certain operations.
%%!!>>

\subsubsection{1-Click Ops}\label{subsubsec:main:1-click-ops} %%##pages_mainpage-1-click-ops==title>>
%%!!pages_mainpage-1-click-ops<<
\hyperref[sec:1-click-ops-page]{1-Click Ops} stands for \textbf{One-Click Operations}. You can directly open the
corresponding page by clicking on this option.
%%!!>>

\subsubsection{App Usage} %%##pages_mainpage-appusage==title>>
%%!!pages_mainpage-appusage<<
App usage statistics such as \textit{screen time}, \textit{data usage} (both mobile and Wifi), \textit{number of times
an app was opened} can be accessed by clicking on the \textbf{App Usage} option in the menu (requires \textit{Usage
Access} permission).
%%!!>>

\subsubsection{System Config} %%##pages_mainpage-systemconfig==title>>
%%!!pages_mainpage-systemconfig<<
Display various system configurations, whitelists/blacklists included in Android by either OEM/vendor, AOSP or even some
Magisk modules.
%%!!>>

\subsubsection{Running Apps}\label{subsubsec:main:running-apps} %%##pages_mainpage-running-apps==title>>
%%!!pages_mainpage-running-apps<<
A list of running apps or processes can be viewed by clicking on the \textbf{Running Apps} option in the menu (requires
root or \hyperref[sec:adb-over-tcp]{ADB}).
Running apps or processes can also be force-closed or killed within the resultant page.
Logs for each process ID (PID) can also be viewed in the \hyperref[subsubsec:log-viewer]{log viewer}.
%%!!>>

\subsubsection{APK Updater} %%##pages_mainpage-apkupdater==title>>
%%!!pages_mainpage-apkupdater<<
If you have the app \href{https://github.com/rumboalla/apkupdater}{APK Updater} installed, you can use the corresponding
option in the menu to open the app directly.
The option is hidden if you do not have it installed.
%%!!>>

\subsubsection{Profiles} %%##pages_mainpage-profiles==title>>
%%!!pages_mainpage-profiles<<
Opens the \hyperref[sec:profiles-page]{profiles page}.
Profiles are a way to configure regularly used tasks.
You can also add shortcuts of the profiles in the launcher to run them directly.
%%!!>>

\subsubsection{Log Viewer}\label{subsubsec:log-viewer} %%##pages_mainpage-log-viewer==title>>
%%!!pages_mainpage-log-viewer<<
Display, manage logs from \texttt{logcat}.
For no-root users, this only logs the activities of App Manager due to its limitation.
However, it can display logs from all users if \texttt{android.permission.READ\_LOGS} is granted using ADB (granted by
automatically when using ADB over TCP).
%%!!>>

\subsubsection{Termux} %%##pages_mainpage-termux==title>>
%%!!pages_mainpage-termux<<
If you have \href{https://github.com/termux/termux-app}{Termux} installed, you can directly go to the running session
or open a new session using the \textbf{Termux} option in the menu.
%%!!>>

\subsubsection{Settings} %%##pages_mainpage-settings==title>>
%%!!pages_mainpage-settings<<
You can go to in-app \hyperref[sec:settings-page]{Settings} by clicking on the corresponding option.
%%!!>>
