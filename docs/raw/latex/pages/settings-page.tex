\section{Settings Page}\label{sec:settings-page}
Settings can be used to customise the behaviour of the app.

\subsection{Language}\label{subsec:language}
Configure in-app language. App Manager currently supports 15 (fifteen) languages.

\subsection{App Theme}\label{subsec:app-theme}
Configure in-app theme.

\subsection{Mode of Operation}\label{subsec:mode-of-operation}
You can select one of the four options:
\begin{itemize}
    \item \textbf{Auto.} Let AM decide the suitable option for you. Although this is the default option, it may cause
    problems in some devices.
    \item \textbf{Root.} Select root mode. By default, AM requests root permission if root is detected but hasn't
    been granted. If this option is selected, AM will run in root mode even if root is unavailable or denied. This may
    cause crashes or freezing issues, therefore, shouldn't be enabled if root is unavailable.
    \item \textbf{ADB over TCP.} Enable \hyperref[sec:adb-over-tcp]{ADB over TCP} mode. AM may hang indefinitely if ADB
    over TCP is not enabled.
    \item \textbf{No-root.} AM runs in no-root mode. AM performs better if this is enabled but all the root/ADB
    features will be disabled.
\end{itemize}

\subsection{Usage Access}\label{subsec:usage-access}
Turning off this option disables the \textbf{App Usage} page as well as \textit{data usage} and \textit{app storage info} in
the \hyperref[subsec:app-info-tab]{App Info tab}. With this option turned off, App Manager will never ask for
\textit{Usage Access} permission.

\subsection{APK Signing}\label{subsec:apk-signing}

\subsubsection{Signature schemes}
Select the \href{https://source.android.com/security/apksigning}{signature schemes} to use. It is recommended that you
use at least v1 and v2 signature schemes. Use the \textit{Reset to Default} button in case you're confused.

\subsection{Rules}\label{subsec:rules}

\subsubsection{Global Component Blocking}\label{subsubsec:instant-component-blocking}
By default, blocking rules are not applied unless they are applied explicitly in \hyperref[sec:app-details-page]{the App
Details page} for any package. Upon enabling this option, all (old and new) rules are applied immediately for all apps
without explicitly enabling blocking for any app.

\begin{warning}{Notice}
    Enabling this setting may have some unintended consequences, such as rules that are not completely removed will be
    applied again. So, proceed with caution. This option should be kept disabled if not required for some reasons.
\end{warning}

\seealsoinline{\hyperref[subsec:faq:what-is-global-component-blocking]{FAQ: What is global component blocking?}}

\subsubsection{Import/Export Blocking Rules}
It is possible to import or export blocking rules within App Manager for all apps. There is a choice to export or import
only certain rules (components, app ops or permissions) instead of all of them. It is also possible to import blocking
rules from \href{https://github.com/lihenggui/blocker}{Blocker} and \href{https://github.com/tuyafeng/Watt}{Watt}. If it
is necessary to export blocking rules for a single app, use the corresponding \hyperref[sec:app-details-page]{App
Details page} to export rules, or for multiple apps, use \hyperref[subsec:batch-operations]{batch operations}.

\seealsoinline{\hyperref[sec:rules-specification]{Rules Specification}}

\paragraph{Export} Export blocking rules for all apps configured within App Manager. This may include
\hyperref[subsec:faq:what-are-app-components]{app components}, app ops and permissions based on the options selected in
the multi-choice options.

\paragraph{Import} Import previously exported blocking rules from App Manager. Similar to export, this may include
\hyperref[subsec:faq:what-are-app-components]{app components}, app ops and permissions based on the options selected in
the multi-choice options.

\paragraph{Import Existing Rules}
\phantomsection
\label{par:import-existing-rules}
Add components disabled by other apps to App Manager. App Manager only keeps track of component disabled within
App Manager. If you use other tools to block app components, you can use this tools to import these disabled components.
Clicking on this option triggers a search for disabled components and will lists apps with components disabled by user.
For safety, all the apps are unselected by default. You can manually select the apps in the list and re-apply the
blocking through App Manager.

\begin{danger}{Caution}
    Be careful when using this tool as there can be many false positives. Choose only the apps that you are certain about.
\end{danger}

\paragraph{Import from Watt} Import configuration files from \href{https://github.com/tuyafeng/Watt}{Watt}, each file
containing rules for a single package and file name being the name of the package with \texttt{.xml} extension.

\begin{tip}{Tip}
    Location of configuration files in Watt: \texttt{/sdcard/Android/data/com.tuyafeng.watt/files/ifw}
\end{tip}

\paragraph{Import from Blocker} Import blocking rules from \href{https://github.com/lihenggui/blocker}{Blocker}, each
file containing rules for a single package. These files have a \texttt{.json} extension.

\subsubsection{Remove all rules}
One-click option to remove all rules configured within App Manager. This will enable all blocked components, app ops
will be set to their default values and permissions will be granted.

\subsection{Installer}\label{subsec:installer}
Installer specific options

\subsubsection{Show users in installer}
For root/ADB users, a list of users will be displayed before installing the app. The app will be installed only for the
specified user (or all users if selected).

\subsubsection{Sign APK}
Whether to sign the APK files before installing the app. See \hyperref[subsec:apk-signing]{APK signing} section above to
configure signing.

\subsubsection{Install location}
Choose APK install location. This can be one of \textit{auto}, \textit{internal only} and \textit{prefer external}.
In newer Android versions, the last option may not always install the app in the external storage.

\subsubsection{Installer App}
Select the installer app, useful for some apps which explicitly checks for their installer. This only works for
root/ADB users.

\subsection{Backup/Restore}\label{subsec:backup/restore}
Settings related to \hyperref[sec:backup-restore]{backup/restore}.

\subsubsection{Compression method}
Set which compression method to be used during backups. App Manager supports GZip and BZip2 compression methods, GZip
being the default compression method. It doesn't affect the restore of an existing backup.

\subsubsection{Backup Options}\label{subsubsec:settings-backup-options}
Customise the backup/restore dialog.

\subsubsection{Backup apps with Android KeyStore}
Allow backup of apps that has entries in the Android KeyStore (disabled by default). Some apps (such as Signal) may
crash if restored. KeyStore backup also doesn't work from Android 9 but still kept as many apps having KeyStore can be
restored without problem.

\subsubsection{Encryption}\label{subsubsec:settings-encryption}
Set an encryption method for the backups. AM currently supports OpenPGP only.

\begin{warning}{Caution}
    In v2.5.16, App Manager doesn't remember key IDs for a particular backup. You have to remember them yourself. This has been fixed in v2.5.18.
\end{warning}

\subsection{Device Info}\label{subsec:device-info}
Display Android version, security, CPU, GPU, battery, memory, screen, languages, user info, etc.
