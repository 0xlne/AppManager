\section{Profile Page}\label{sec:profile-page}
Profile page displays the configurations for a profile. It also offers editing them.

\begin{tip}{Notice}
    When you apply a profile, if some packages do not match the criteria, they will simply be ignored.
\end{tip}

\subsection{Options Menu}\label{subsec:profile-options-menu}
The three dots menu on the top-right corner opens the options-menu. It contains several options such as--
\begin{itemize}
    \item \textbf{Apply.} This option can be used to apply the profile. When clicked, a dialog will be displayed where
    you can select a \hyperref[subsubsec:profile-state]{profile state}. On selecting one of the options, the profile
    will be applied immediately.
    \item \textbf{Save.} Allows you to save the profile.
    \begin{warning}{Notice}
        Changes are never saved automatically. You have to save them manually from here.
    \end{warning}
    \item \textbf{Discard.} Discard any modifications made since the last save.
    \item \textbf{Delete.} Clicking on delete will remove the profile immediately without any warning.
    \item \textbf{Duplicate.} This option can be used to duplicate the profile. When clicked, an input box will be
    displayed where you can set the profile name. If you click ``OK'', a new profile will be created, and the page will be
    reloaded. The profile will not be saved until you save it manually.
    \item \textbf{Create shortcut.} This option can be used to create a shortcut for the profile. When clicked, there
    will be two options: \textit{Simple} and \textit{Advanced}. The latter option allows you to set the profile state
    before applying it while the former option uses the default state that was configured when the profile was last saved.
\end{itemize}

\subsection{Apps Tab}\label{subsec:profile-apps-tab}
Apps tab lists the packages configured under this profile. Packages can be added or removed using the \textit{plus}
button located near the bottom of the screen. Packages can also be removed by long clicking on them (in which case, a
popup will be displayed with the only option \textit{delete}).

\subsection{Configurations Tab}\label{subsec:profile-configurations-tab}
Configurations tab can be used to configure the selected packages. Description of each item is given below:

\subsubsection{Comment}
This is the text that will be displayed in the \hyperref[sec:profiles-page]{profiles page}. If not set, the current
configurations will be displayed instead.

\subsubsection{State}\label{subsubsec:profile-state}
Denotes how certain configured options will behave. For instance, if \textit{disable} option is turned on, the apps will
be disabled if the state is \textit{on} and will be enabled if the state is \textit{off}. Currently state only support
\textit{on} and \textit{off} values.

\subsubsection{Users}
Select users for which is the profile will be applied. All users are selected by default.

\subsubsection{Components}
This behaves the same way as the \hyperref[subsec:block-components-dots]{Block Components\dots} option does in the
1-Click Ops page. However, this only applies for the selected packages. If the \hyperref[subsubsec:profile-state]{state}
is \textit{on}, the components will be blocked, and if the state is \textit{off}, the components will be unblocked.
The option can be disabled (regardless of the inserted values) by clicking on the \textit{disabled} button on the input
dialog.

\seealsoinline{\hyperref[subsec:faq:what-are-app-components]{What are the app components?}}

\subsubsection{App Ops}
This behaves the same way as the \hyperref[subsec:set-mode-for-app-ops-dots]{Set Mode for App Ops\dots} option does in
the 1-Click Ops page. However, this only applies for the selected packages. If the
\hyperref[subsubsec:profile-state]{state} is \textit{on}, the app ops will be denied (ie. ignored), and if the state is
\textit{off}, the app ops will be allowed. The option can be disabled (regardless of the inserted values) by clicking
on the \textit{disabled} button on the input dialog.

\subsubsection{Permissions}
This option can be used to grant or revoke certain permissions from the selected packages. Like others above,
permissions must be separated by spaces. If the \hyperref[subsubsec:profile-state]{state} is \textit{on}, the
permissions will be revoked, and if the state is \textit{off}, the permissions will be allowed. The option can be
disabled (regardless of the inserted values) by clicking on the \textit{disabled} button on the input dialog.

\subsubsection{Backup/Restore}
This option can be used to take a backup of the selected apps and its data or restore them. There two options available
there: \textit{Backup options} and \textit{backup name}.
\begin{itemize}
    \item \textbf{Backup options.} Same as the \hyperref[subsec:backup-restore-backup-options]{backup options} of the
    backup/restore feature. If not set, the default options will be used.
    \item \textbf{Backup name.} Set a custom name for the backup. If the backup name is set, each time a backup is made,
    it will be given a unique name with backup-name as the suffix. This behaviour will be fixed in a future release.
    Leave this field empty for regular or ``base'' backup (also, make sure not to enable \textit{backup multiple} in the
    backup options).
\end{itemize}

If the \hyperref[subsubsec:profile-state]{state} is \textit{on}, the packages will be backed up, and if the state is
\textit{off}, the packages will be restored. The option can be disabled by clicking on the \textit{disabled} button on
the input dialog.

\subsubsection{Export Blocking Rules}
This option allows you to export blocking rules.

\begin{danger}{Danger}
    This option is not yet implemented.
\end{danger}

\subsubsection{Disable}
Enabling this option will enable/disable the selected packages depending on the \hyperref[subsubsec:profile-state]{state}.
If the state is \textit{on}, the packages will be disabled, and if the state is \textit{off}, the packages will be enabled.

\subsubsection{Force-stop}
Enabling this option will allow the selected packages to be force-stopped.

\subsubsection{Clear Cache}
Enabling this option will enable clearing cache for the selected packages.

\subsubsection{Clear Data}
Enabling this option will enable clearing data for the selected packages.

\subsubsection{Block Trackers}
Enabling this option will block/unblock tracker components from the selected packages depending on the
\hyperref[subsubsec:profile-state]{state}. If the state is \textit{on}, the trackers will be blocked, and if the state
is \textit{off}, the trackers will be unblocked.

\subsubsection{Backup APK}
Enabling this option will enable APK backup for the selected packages. This is not the same as
\hyperref[sec:backup-restore]{backup/restore} as described there.
