\section{1-Click Ops Page}\label{sec:1-click-ops-page}
This page appears after clicking on the \hyperref[subsubsec:main:1-click-ops]{1-Click Ops option} in the
\hyperref[subsec:main-page-options-menu]{main menu}.

\subsection{Block/Unblock Trackers}\label{subsec:block-unblock-trackers}
This option can be used to block or unblock the ad/tracker components from the installed apps. After you click on this
option, you will be asked to select if AM will list trackers from all apps or only from the user apps. Novice users
should avoid blocking trackers from the system apps in order to avoid consequences. After that, a multi-choice dialog
box will appear where you can deselect the apps you want to exclude from this operation. Clicking \textit{block} or
\textit{unblock} applies the changes immediately.

\begin{warning}{Notice}
    Certain apps may not function as expected after applying the blocking. If that is the case, remove blocking rules
    all at once or one by one in the component tabs of the corresponding \hyperref[sec:app-details-page]{App Details page}.
\end{warning}

\begin{amseealso}
    \item \hyperref[subsec:faq:how-to-unblock-tracker-components]{How to unblock the tracker components blocked using 1-Click Ops or Batch Ops?}
    \item \hyperref[par:appdetails:blocking-trackers]{App Details Page: Blocking Trackers}
\end{amseealso}

\subsection{Block Components\dots}\label{subsec:block-components-dots}
This option can be used to block certain app components denoted by the signatures. App signature is the full name or
partial name of the components. For safety, it is recommended that you should add a \texttt{.} (dot) at the end of each
partial signature name as the algorithm used here chooses all matched components in a greedy manner. You can insert more
than one signature in which case all signatures have to be separated by spaces. Similar to the option above, there is an
option to apply blocking to system apps as well.

\begin{danger}{Caution}
    If you are not aware of the consequences of blocking app components by signature(s), you should avoid using this
    setting as it may result in boot loop or soft brick, and you may have to apply factory reset in order to use your OS\@.
\end{danger}

\subsection{Set Mode for App Ops\dots}\label{subsec:set-mode-for-app-ops-dots}
This option can be used to configure certain \hyperref[ch:app-ops]{app operations} of all or selected apps. You can
insert more than one app op constants separated by spaces. It is not always possible to know in advance about all the
app op constants as they vary from device to device and from OS to OS. To find the desired app op constant, browse the
\textit{App Ops} tab in the \hyperref[sec:app-details-page]{App Details page}. The constants are integers closed inside
brackets next to each app op name. You can also use the app op names. You also have to select one of the
\hyperref[subsec:mode-constants]{modes} that will be applied against the app ops.

\begin{danger}{Caution}
    Unless you are well-informed about app ops and the consequences of blocking them, you should avoid using this
    feature as it may result in boot loop or soft brick, and you may have to apply factory reset in order to use your OS\@.
\end{danger}
