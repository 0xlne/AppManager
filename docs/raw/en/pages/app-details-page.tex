% SPDX-License-Identifier: GPL-3.0-or-later OR CC-BY-SA-4.0
\section{App Details Page}\label{sec:app-details-page} %%##$section-title>>
%%!!intro<<
\textbf{App Details} page consists of 11 (eleven) tabs. It describes almost every bit of information an application
can have, including all attributes from its manifest, \hyperref[ch:app-ops]{application operations}, signing
information, libraries, and so on.
%%!!>>

\subsection{Colour Codes}\label{subsec:app-details-colour-codes} %%##$colour-codes-title>>
%%!!colour-codes<<
List of colours used in this page, and their meaning:
\begin{itemize}
    \item \colorbox{AMRed}{\textcolor{black}{Red (day)}} / \colorbox{AMDarkRed}{\textcolor{white}{dark red (night)}}
    -- Denotes any app ops or permissions having the dangerous flag, or any components blocked within App Manager

    \item \colorbox{AMLightRed}{\textcolor{black}{Light red (day)}} / \colorbox{AMVeryDarkRed}{\textcolor{white}{very
    dark red (night)}} -- Denotes the components disabled outside of App Manager

    \begin{tip}{Note}
        A component marked as disabled does not always mean that it is disabled by the user: It could also be disabled
        by the system or marked as disabled in its manifest. The components of a disabled application are also
        considered disabled by the system (and App Manager).
    \end{tip}

    \item \colorbox{AMVividOrange}{\textcolor{black}{Vivid orange (day)}} / \colorbox{AMVeryDarkOrange}{
        \textcolor{white}{very dark orange (night)}} -- Denotes the tracker components

    \item \colorbox{AMSoftMagenta}{\textcolor{black}{Soft magenta (day)}} / \colorbox{AMVeryDarkViolet}{
        \textcolor{white}{very dark violet (night)}} -- Denotes the running services.
\end{itemize}
%%!!>>

\subsection{App Info Tab}\label{subsec:app-info-tab} %%##$app-info-tab-title>>
%%!!app-info-tab<<
\textbf{App Info} tab contains general information about an application. It also lists many actions that can be
performed within this tab.
%%!!>>

\subsubsection{General Information}\label{subsubsec:app-info-general-information} %%##$app-info-general-information-title>>
%%!!app-info-general-information<<
The list below is in the same order as listed in the App Info tab.
\begin{itemize}
    \item \textbf{Application Icon.} The application icon. If the application does not have an icon, the system default
    icon will be displayed. It is also possible to verify the APK signature via SHA or MD5 sums stored in the clipboard
    by simply clicking on it.

    \item \textbf{Application Label.} The application label or the name of the application.

    \item \textbf{Package Name.} The name of the application package. Clicking on the name stores it in the clipboard.

    \item \textbf{Version.} The application version is divided into two parts. The first part is called \textit{version
    name}. The format of this part varies but often consists of multiple integers separated by dots. The second part
    is called \textit{version code}. It is enclosed by the first brackets. The version code is an integer used to
    differentiate between application versions (since a version name can be unreadable to a machine). In general,
    a new version of an application has higher version code than the old ones. For example, if \texttt{123} and
    \texttt{125} are two version codes of an application, we can say that the latter is more updated than the former
    because the version code of the latter is higher. Applications that serve different APK files for the same version
    on different platforms (mobile, tabs, desktops, etc.) or architectures (32/64 bit, ARM or Intel), the version
    numbers can be misleading as they often add prefixes for each platform.

    \item \textbf{Tags.} (Also known as tag clouds) Tags include the most basic, concise and useful information of an
    application, such as---
    \begin{itemize}
        \item \textit{Tracker info.} Number of tracker components in the application (e.g., \textit{5 trackers}) The
        colour of the tag appears orange if the trackers are unblocked and dark cyan if they are blocked in App Manager.
        Clicking on the tag opens a dialog containing the list of tracker components which can be blocked or unblocked
        if App Manager has sufficient privileges.
        \item \textit{Application type.} User application or system application. If it is a system application, whether
        the application is an updated version of the system application or if the application is installed systemless-ly
        via Magisk.
        \item \textit{Split APK info.} Number of splits in the APK excluding the base APK (e.g., \textit{5 splits}).
        Clicking on the tag opens a dialog containing the split APK information such as type and size.
        \item \textit{Debuggable.} The application can be debugged over ADB\@. Debuggable applications can enjoy certain
        functions unavailable to a regular application. The data of the application might be access via ADB (e.g.\ using
        \texttt{run-as} command) without any additional permissions.
        \item \textit{Test only.} The application is a test-only application. Test-only applications can enjoy certain
        functions unavailable to a regular application. The data of the application might be access via ADB (e.g.\ using
        \texttt{run-as} command) without any additional permissions.
        \item \textit{Large heap.} The application has requested large heap size i.e.\ more space in memory (RAM) is
        requested for dynamic allocation. It is still up to the operating system to decide whether to allocate large
        space for the application. App Manager, for example, requests large heap size because it needs to load an entire
        APK into memory while scanning an APK before Android 8.
        \item \textit{No code.} The application does have any code associated with it i.e.\ DEX files aren't present.
        In some system applications, the actual code might be located in another place.
        \item \textit{Running.} One or more services of the application is currently running in the background. Clicking
        on the tag opens a dialog containing the list of running services. Clicking on any services opens it in the log
        viewer provided the log viewer feature is enabled.
        \item \textit{Stopped.} The application is force stopped. This may not prevent it from running automatically
        later.
        \item \textit{Disabled.} Denotes that the application is disabled (hidden from the launcher).
        \item \textit{Suspended.} Denotes that the application is suspended (grayed out in the launcher).
        \item \textit{Hidden.} Denotes that the application is hidden (hidden from the launcher).
        \item \textit{MagiskHide.} MagiskHide is enabled. Clicking on the tag opens a dialog containing the list of
        processes within the application that can be added or removed from the MagiskHide list.
        \item \textit{MagiskDenyList.} The application is present in MagiskDenyList. Clicking on the tag opens a dialog
        containing the list of processes within the application that can be added or removed from MagiskDenyList.
        \item \textit{KeyStore.} The application has items in the Android KeyStore. Clicking on the tag opens a dialog
        containing all the KeyStore files that belong to the application.
        \item \textit{Backup.} The application was backed up using App Manager at least once. Clicking on the tag
        opens a dialog containing all the available backups along with metadata.
        \item \textit{No battery optimisation.} Battery optimisation is disabled for the application. It is possible to
        re-enable battery optimisation by clicking on the tag.
        \item \hyperref[sec:net-policy]{\textit{Net policy.}} Network policy (e.g., background data usage) is configured
        for the application. Clicking on the tag displays a dialog containing the supported policies for the platform
        along with the options to configure them.
        \item \hyperref[sec:terminologies]{\textit{SSAID.}} Clicking on the tag opens a dialog containing the current
        SSAID assigned to the application. It is also possible to reset/regenerate the SSAID if needed.
        \item \textit{SAF.} Denotes that the application has been granted to access one or more storage locations or
        files i.e.\ URIs via Storage Access Framework (SAF). Clicking on the tag opens a dialog containing the list of
        granted URIs.
        \item \textit{Play App Signing.} Indicates that the application might be signed by Google.
    \end{itemize}

    \item \textbf{Horizontal Action Panel.} An action panel consisting of various actions that can be carried out for
    the application. See §\ref{subsubsec:horizontal-action-panel} for a complete list of actions available here.
    Additional actions are available in the \hyperref[subsubsec:app-info-options-menu]{options menu}.

    \item \textbf{Paths \& Directories.} Contains various information regarding application paths including \textit{app
    directory} (where the APK files reside), \textit{data directories} (internal, device protected and externals),
    \textit{split APK directories} (along with the split names), and \textit{native JNI library} (if present). JNI
    libraries are used to invoke native codes usually written in C/C++. Use of native library can make the application
    run faster or help an application use third-party libraries written using languages other than Java like in most
    games. The directories can be opened via third-party file managers provided they support it and have the necessary
    permissions by clicking on the launch icon on the right-hand side of each directory item.

    \item \textbf{Data Usage.} Amount of data used by the application as reported by the operating system. Depending on
    Android version, this may require a wide range of permissions including \textit{Usage Access} and \textit{Telephony}
    permissions.

    \item \textbf{Storage \& Cache.} Displays information regarding the size of the application (APK files, optimised
    files), data and cache. In older devices, size of external data, cache, media and OBB folders are also displayed.
    This part remains hidden if\textit{Usage Access} permission is not granted in newer devices.

    \item \textbf{More Info.} Displays other information such as--
    \begin{itemize}
        \item \textbf{SDK.} Displays information related to the Android SDK. There are two (one in old devices) values:
        \textit{Max} denotes the target SDK and \textit{Min} denotes the minimum SDK (the latter is not available in old
        devices). It is best practice to use applications with maximum SDK that the platform currently supports in order
        to ensure that the application is not running in the compatibility mode in order to use privacy-evading features.
        SDK is also known as \textbf{API Level}.
        \seealsoinline{\href{https://en.wikipedia.org/wiki/Android_version_history\#Overview}{Android Version History}}

        \item \textbf{Flags.} The application flags used at the time of building the application. For a complete list of
        flags and what they do, visit the
        \href{https://developer.android.com/reference/android/content/pm/ApplicationInfo\#flags}{official documentation}.

        \item \textbf{Date Installed.} The date when the application was first installed.

        \item \textbf{Date Updated.} The date when the application was last updated. This is the same as \textit{Date Installed}
        if the application hasn't been updated.

        \item \textbf{Installer App.} The application that installed this application. Not all application supply the information used by the
        package manager to register the installer application. Therefore, this value should not be taken for granted.

        \item \textbf{User ID.} The unique user ID set by the Android system to the application. For shared applications, same
        user ID is assigned to multiple applications that have the same \textit{Shared User ID}.

        \item \textbf{Shared User ID.} Applicable for applications that are shared together. The shared application must
        have the same \hyperref[subsec:signatures-tab]{signatures}.

        \item \textbf{Primary ABI.} Architecture supported by this platform for this application.

        \item \textbf{Zygote preload name.} Responsible for preloading application code and data shared across all the
        isolated services that uses app zygote.

        \item \textbf{Hidden API enforcement policy.} Since Android 9, many methods and classes in Android framework
        have been made inaccessible to the third-party applications through hidden API enforcement policy. It has the
        following options:
        \begin{itemize}
            \item \textit{Default.} Based on the type of application. For system applications, it should be disabled,
            and for others, it should be enforced.
            \item \textit{None/disabled.} The application has full access to the hidden API as it used to be before
            Android 9.
            \item \textit{Warn.} Same as above, except that warnings will be logged each time the application accesses
            the hidden API. This is mostly unused.
            \item \textit{Enforce.} The application cannot access hidden API, either dark-grey list or blacklist, or
            both of them. This is the default option for the third-party applications in Android 9 and later unless the
            application is whitelisted by the OEM or the vendor.
            \begin{warning}{Warning}
                Hidden API enforcement policy is not properly implemented in Android and can be bypassed by the
                application. As a result, this value should not be trusted.
            \end{warning}
        \end{itemize}

        \item \textbf{SELinux.} Mandatory access control (MAC) policy set by the operating system via SELinux.

        \item \textbf{Main Activity.} The main entry point to the application. This is only visible if the application has
        \hyperref[subsubsec:activities]{activities} and any of those are openable from the Launcher. There's also a
        launch button on the right-hand side which can be used to launch this activity.
    \end{itemize}
\end{itemize}
%%!!>>

\subsubsection{Horizontal Action Panel}\label{subsubsec:horizontal-action-panel} %%##$horizontal-action-panel-title>>
%%!!horizontal-action-panel<<
Horizontal Action Panel, as described in the previous section, consists of various application-related actions, such as--
\begin{itemize}
    \item \textbf{Launch.} Launch the application provided it has a launcher \hyperref[subsubsec:activities]{activity}.

    \item \textbf{Disable.} Disable the application. This button is not displayed if it is already disabled or the user
    does not have enough privileges. After the application is disabled, it will be hidden from the app drawer. Shortcuts
    for the application may also be removed. The application may be re-enabled only via App Manager or any other tool
    that supports it. There isn't any option in Android Settings to enable a disabled user application.

    \item \textbf{Uninstall.} Uninstall the application.

    \item \textbf{Enable.} Enable the application. This button is not displayed if it is already enabled or the user
    does not have enough privileges.

    \item \textbf{Force Stop.} Force-stop the application.

    \item \textbf{Clear Data.} Clear data from the application. This includes any information stored in the internal
    and, recently, the external directories, including accounts (if set by the application), cache, etc. Clearing data
    from App Manager, for example, removes all the rules (the blocking is not removed though) saved within the
    application. Which is why you should always take backups of your rules. This button is not displayed if the user
    does have enough privileges.

    \item \textbf{Clear Cache.} Clear application cache. There is not any Android-way to clear the cache of a specific
    application. Therefore, it needs root permission to clear the cache of the application's internal storage.

    \item \textbf{Install.} Action to install an application opened via a third-party application. This button is only
    displayed if the application hasn't been installed.

    \item \textbf{What's New.} This button is displayed for an external application if it's already installed. Clicking
    on this button displays a dialog consisting of differences between the opened version and the installed version in a
    version control manner. The information it displays include \textit{version}, \textit{trackers},
    \textit{permissions}, \textit{components}, \textit{signatures} (checksum changes), \textit{features},
    \textit{shared libraries} and \textit{SDK}.

    \item \textbf{Update.} Displayed if the application has a higher version code than the installed application.

    \item \textbf{Reinstall.} Displayed if the application has the same version code as the installed application.

    \item \textbf{Downgrade.} Displayed if the application has a lower version code than the installed application.

    \item \textbf{Manifest.} Clicking on this button displays the application's manifest file in a separate page. The
    manifest file can be wrapped or unwrapped using the corresponding toggle button (on the top-right side) or can be
    saved to the storage using the save button.

    \item \textbf{Scanner.} Scans the application in order to list potential trackers and libraries. It also scans the
    file using VirusTotal if configured.\\
    \seealsoinline{\hyperref[sec:scanner-page]{Scanner page}}

    \item \textbf{Shared Prefs.} Clicking on this button displays a list of shared preferences used by the application.
    Clicking on a preference item in the list opens the \hyperref[sec:shared-preferences-editor-page]{Shared Preferences
    Editor page}. This option is only visible if the user has the required privileges.

    \item \textbf{Databases.} Clicking on this button displays a list of databases used by the application. This option
    is only visible if the user has the required privileges.

    \item \textbf{F-Droid.} Opens the application in your favourite \textit{F-Droid} client.

    \item \textbf{Store.} Opens the application in \textit{Aurora Store}. The option is only visible if \textit{Aurora
    Store} is installed.
\end{itemize}
%%!!>>

\subsubsection{Options Menu}\label{subsubsec:app-info-options-menu} %%##$app-info-options-menu-title>>
%%!!app-info-options-menu<<
Options-menu is located in the top-right corner of the page. A complete description of the options present there are
given below:
\begin{itemize}
    \item \textbf{Share.} Share button can be used to share the APK or (if the application is has multiple splits)
    \textit{APKS} file extracted from the application.

    \item \textbf{Refresh.} Refresh the App Info tab.

    \item \textbf{View in Settings.} Open the application in Android Settings.

    \item \textbf{Backup/restore.} Open the backup/restore dialog.

    \item \textbf{Export blocking rules.} Export rules configured for the application within App Manager.

    \item \textbf{Open in Termux.} Open the application in Termux. This actually runs \texttt{su - user\_id} where
    \texttt{user\_id} denotes the application's kernel user ID (described in §\ref{subsubsec:app-info-general-information}).
    This option is only visible to the root users. See §\ref{subsubsec:config-termux} to learn how to configure Termux
    to run commands from third-party applications.

    \item \textbf{Run in Termux.} Open the application via \texttt{run-as package\_name} in Termux. This is only
    applicable to the debuggable applications and works for both root and ADB users. See §\ref{subsubsec:config-termux}
    to learn how to configure Termux to run commands from third-party applications.

    \item \textbf{MagiskHide.} Open a dialog containing the list of processes within the application that can be added
    or removed from the MagiskHide list.

    \item \textbf{MagiskDenyList.} Open a dialog containing the list of processes within the application that can be
    added or removed from MagiskDenyList.

    \item \textbf{Battery optimisation.} Enable/disable battery optimisation.

    \item \hyperref[sec:net-policy]{\textbf{Net policy.}} Configure network policy (e.g., background data usage) for the
    application.

    \item \textbf{Extract Icon.} Extract and save the application's icon in the shared storage.

    \item \textbf{Add to profile.} Add the application to one of the configured \hyperref[sec:profile-page]{profiles}.
\end{itemize}
%%!!>>

\subsubsection{Configuring Termux}\label{subsubsec:config-termux} %%##$config-termux-title>>
%%!!config-termux<<
By default, Termux does not allow running commands from third-party applications. To use this option, Termux v0.96 or
later is required and \texttt{allow-external-apps=true} must be added in \texttt{\textasciitilde/.termux/termux.properties}.

\begin{tip}{Info}
    Enabling this option does not weaken Termux' security. The third-party applications still need to ask the user to
    allow running arbitrary commands in Termux.
\end{tip}
%%!!>>

\subsection{Component Tabs}\label{subsec:component-tabs} %%##$component-tabs-title>>
%%!!component-tabs<<
\textbf{Activities}, \textbf{Services}, \textbf{Receivers} (i.e. broadcast receivers) and \textbf{Providers}
(i.e. content providers) are collectively known as the application components. This is because they share
similar features in many ways. For example, they all have a \textit{name}, a \textit{label}, an \textit{icon} and are
executed via \textit{Intent}. Application components are the building blocks of an application and must be declared in
the application manifest. Application manifest is a file where application specific metadata are stored. The Android
operating system learns what to do with the application by reading the metadata.

Colours used in these tabs are explained in §\ref{subsec:app-details-colour-codes}. It is also possible to sort the list
of components to display blocked or tracker components on top of the list via the \textbf{Sort} option located in the
overflow menu.
%%!!>>

\subsubsection{Activities}\label{subsubsec:activities} %%##$activities-title>>
%%!!activities<<
\textbf{Activities} are the windows or pages that can be uniquely identified by the Android operating system (e.g.,
\textit{Main page} and \textit{App Details page} are two activities). Each activity can have multiple UI components
known as \textit{widgets} or \textit{fragments}, and each component can be nested or placed on top of each other. The
developer can also choose to open external files, links, etc. within an activity using a method called
\textit{intent filters}. For example, when you open a file using your file manager, either your file manager or
operating system scans the intent filters via PackageManager to find the activities capable of opening the file and
offers you to open the file with these activities.

Activities which are \textit{exportable} can usually be opened by any third-party applications. Some activities require
permissions, and if that is the case, only an application having those permissions can open them. In the
\textit{Activities} tab, activities can be launched via the \textbf{Launch} button. If it is necessary to supply
additional information such as Intent extras, data or action, long clicking on the \textbf{Launch} button opens the
\hyperref[sec:interceptor-page]{Activity Interceptor} page which provides such features.

\begin{tip}{Notice}
    If you are unable to open any activity, chances are it has certain dependencies which are not met, e.g.\ you
    cannot open \textit{App Details} page because it requires you to, at least, supply a package name. Since these
    dependencies cannot be inferred programmatically, these activities may not be opened via App Manager by default.
\end{tip}

It is also possible to create shortcuts of the activities via the \textbf{Create shortcut} button.

\begin{danger}{Caution}
    If you uninstall App Manager, all shortcuts created by App Manager will be lost.
\end{danger}
%%!!>>

\subsubsection{Services}\label{subsubsec:details:servcies} %%##$servcies-title>>
%%!!servcies<<
Unlike \hyperref[subsubsec:activities]{activities} that users can see, \textbf{Services} handle background tasks. For example,
if you're downloading a video from the internet using your phone's Internet browser, the Internet browser is using a
\textit{foreground service} to download the content.

When an activity is closed or removed from the \textit{Recents} section, it may be destroyed immediately depending on
many factors such as how much free memory the phone has. But services can be run indefinitely if desired. If more
services are run in background, the phone may become slower due to the shortage of memory and/or processing power, and
the phone's battery will be drained more quickly. Newer Android versions have a battery optimisation feature enabled
by default for all applications. With this feature enabled, the system can randomly terminate any service. However,
foreground services (i.e. services that runs with a fixed notification such as music player) are not typically
terminated unless the system is low on resources (memory, battery, etc.). Vendor-specific stock ROMs can offer more
aggressive optimisation. MIUI, for example, has a very aggressive optimisation feature known as \textit{MIUI optimisation}.

Both activities and services are run in the same \href{https://stackoverflow.com/questions/7597742}{looper} called
the main looper, which means the services are not really run in the background. It is the task of the developer to
ensure this. How do the application communicate with the service? It uses
\hyperref[subsubsec:app-details-receivers]{broadcast receiver} or Binder.
%%!!>>

\subsubsection{Receivers}\label{subsubsec:app-details-receivers} %%##$receivers-title>>
%%!!receivers<<
\textbf{Receivers} (also called \textit{broadcast receivers}) can be used to trigger execution of certain tasks when
certain events occur. These components are called broadcast receivers because they are executed as soon as a broadcast
message is received. These broadcast messages are sent using a method called \textit{Intent}. Intent is a special
feature in Android that can be used to open applications (i.e.\ activities), run services and send broadcast messages.
Therefore, like \hyperref[subsubsec:activities]{activities}, broadcast receivers use \textit{intent filters} to receive
the desired broadcast messages. Broadcast messages can be sent by the system or the application itself. When a broadcast
message is sent, the corresponding receivers are activated by the system so that they can execute tasks. For example, if
your phone is low on resources, it may freeze or experience lags for a moment after you enable mobile data or connect it
to the Wi-Fi. This is because broadcast receivers that can receive \texttt{android.net.conn.CONNECTIVITY\_CHANGE} are
activated by the system as soon as the data connection is enabled. Since many applications typically use this intent
filter, they are all activated almost immediately by the system which causes the freezing or lags.

Receivers can also be used for inter-process communication (IPC), i.e.\ it can be used to communicate acrosss multiple
applications or even different components of a single application.
%%!!>>

\subsubsection{Providers}\label{subsubsec:providers} %%##$providers-title>>
%%!!providers<<
\textbf{Providers} (also called \textit{content providers}) are used for data management. For example, when you save an
APK file or export rules in App Manager, it uses a content provider called \texttt{.fm.FmProvider} to save the APK or
export the rules. There are many content providers, including the ones provided by the system, to manage various
content-related tasks such as database management, tracking, searching, etc. Each content provider has a field called
\textit{Authority} which is unique to the application in the entire Android ecosystem just as the package name.
%%!!>>

\subsubsection{Additional Features for Rooted Phones} %%##$additional-features-for-rooted-phones-title>>
%%!!additional-features-for-rooted-phones<<
Unlike the no-root users who are mostly spectators in these tabs, root users can perform various operations.
%%!!>>

\paragraph{Blocking Components} %%##$blocking-components-title>>
%%!!blocking-components<<
On the right-most side of each component item, there is a ``block'' icon (which becomes a ``unblock/restore'' icon when
the component is being blocked). This icon (actually, a button) can be used to toggle the blocking status of that
particular component. If \hyperref[subsubsec:instant-component-blocking]{Instant Component Blocking} is not enabled or
blocking is never applied to the application before, it is required to apply the changes using the \textbf{Apply rules}
option in three-dots menu. It is also possible to remove the already-applied rules using the same option (which would be
read as \textbf{Remove rules} this time).

It is also possible to block the component using one of the several method by long clicking on the icon.

\seealsoinline{\hyperref[sec:faq:app-components]{FAQ: App Components}}
%%!!>>

\paragraph{Blocking Trackers}\label{par:appdetails:blocking-trackers} %%##$blocking-trackers-title>>
\phantomsection
%%!!blocking-trackers<<
It is possible to disable tracker components using the \textbf{Block tracker} option in the three-dots menu. All tracker
components will be blocked regardless of the tab you're currently in.

\begin{tip}{Info}
    Tracker components are a subset of application components. Therefore, they are blocked using the same method used
    for blocking any other components.
\end{tip}

\begin{amseealso}
    \item \hyperref[subsec:tracker-classes-versus-tracker-components]{FAQ: Tracker classes versus tracker components}
    \item \hyperref[sec:scanner-page]{Scanner Page}
    \item \hyperref[subsec:block-unblock-trackers]{1-Click Ops Page: Block/Unblock Trackers}
\end{amseealso}
%%!!>>

\subsection{Permission Tabs}\label{subsec:permission-tabs} %%##$permission-tabs-title>>
%%!!permission-tabs<<
\textbf{App Ops}, \textbf{Uses Permissions} and \textbf{Permissions} tabs are related to permissions. In Android
communication across applications or processes which do not have the same identity (known as \textit{shared ID}) often
require permissions. These permissions are managed by the permission controller. Some permissions are considered
\textit{normal} permissions which are granted automatically if they appear in the application manifest, but
\textit{dangerous} and \textit{development} permissions require confirmation from the user. Colours used in these tabs
are explained in §\ref{subsec:app-details-colour-codes}.
%%!!>>

\subsubsection{App Ops}\label{subsubsec:app-ops} %%##$app-ops-title>>
%%!!app-ops<<
\textbf{App Ops} stands for \textbf{Application Operations}. Since Android 4.3, \textit{App Ops} are used by Android to
control many system permissions. Each app op has a unique number associated with it which is displayed along with the
private name of the operation in the App Ops tab. Some app ops also have a public name. A large number of app ops are
also associated with \textit{permissions}. In this tab, an app op is considered dangerous if its associated permission
is marked as dangerous. Other information such as \textit{flags}, \textit{permission name}, \textit{permission description},
\textit{package name}, \textit{group} are also taken from the associated \hyperref[subsubsec:permissions]{permission}.
Others may include the following:
\begin{itemize}
    \item \textbf{Mode.} It describes the current authorisation status which can be \textit{allow}, \textit{deny} (a
    rather misnomer, it simply means error), \textit{ignore} (it actually means deny), \textit{default} (inferred from
    a list of defaults set internally by the vendor or the AOSP), \textit{foreground} (in newer Android versions, it
    means the app op can only be used when the application is running in foreground), and some custom modes set by the
    vendors (MIUI uses \textit{ask}, for example).

    \item \textbf{Duration.} The amount of time this application op has been used (there can be negative durations whose
    use cases are currently unknown to me).

    \item \textbf{Accept Time.} Last time the app op was accepted.

    \item \textbf{Reject Time.} Last time the app op was rejected.
\end{itemize}

\begin{tip}{Info}
    Contents of this tab are visible to no-root users if \texttt{android.permission.GET\_APP\_OPS\_STATS} is granted via ADB\@.
\end{tip}

There is a toggle button next to each app op item which can be used to allow or deny (ignore) it. Other supported modes
can also be set by long clicking on the toggle button. If the desired app op is not listed in the tab,
\textit{Set custom app op} option in the menu can be used instead. It is also possible to reset the changes
using the \textit{Reset to default} option, or deny all the dangerous app ops using the corresponding option in the menu.
Due to the nature how app ops work, the system may take some time to apply them.

\begin{tip}{Tip}
    Denying certain app ops may cause the application to misbehave. If all attempts fail, \textit{reset to default}
    option can be used as the last resort.
\end{tip}

It is possible to sort the list in ascending order by app op names and the associated unique numbers (or values), or
list the denied app ops first using the corresponding sorting options.

\seealsoinline{\hyperref[ch:app-ops]{Appendix: App Ops}}
%%!!>>

\subsubsection{Uses Permissions} %%##$uses-permissions-title>>
%%!!uses-permissions<<
\textbf{Uses Permissions} are the permissions used by the application. This is named so because they are specified in
the manifest using \texttt{uses-permission} tags. Information such as \textit{flags}, \textit{permission name},
\textit{permission description}, \textit{package name}, \textit{group} are taken from the associated
\hyperref[subsubsec:permissions]{permission}.

Privileged users can grant or revoke the \textit{dangerous} and \textit{development} permissions via the toggle button
on the right side of each permission item. It is also possible revoke dangerous permissions all at once using the
corresponding option in the menu. Only these two types of permissions can be revoked because Android does not allow
the modification of \textit{normal} permissions (which most of them are). It might still be possible to revoke them
by editing \texttt{runtime-permissions.xml} itself, but whether this is a possibility is still being investigated.

\begin{tip}{Info}
    Since dangerous permissions are revoked by default by the system, revoking all dangerous permissions is the same as
    resetting all the permissions.
\end{tip}

It is possible to sort the permissions by their name (in ascending order) or choose to display denied or dangerous
permissions at first using the corresponding options in the menu.
%%!!>>

\subsubsection{Permissions}\label{subsubsec:permissions} %%##$permissions-title>>
%%!!permissions<<
\textbf{Permissions} are usually custom permissions defined by the application itself. These type of permissions are
marked as \textit{Internal} permissions. It also contains permissions declared in other applications which are marked as
\textit{External} permissions. External permissions are specified in the application components as the dependencies i.e.
an application may invoke the component only if holds the specified permission. Here's a complete description of each
item that is displayed there:
\begin{itemize}
    \item \textbf{Name.} Each permission has a unique name like \texttt{android.permission.INTERNET} but multiple
    applications can request the same permission.

    \item \textbf{Icon.} Each permission can have a custom icon. The other permission tabs do not have any icon because
    they do not contain any icon in the application manifest.

    \item \textbf{Description.} This optional field describes the permission. If there isn't any description associated
    with the permission, the field is not displayed.

    \item \textbf{Flags.} (Uses the flag symbol or \textbf{Protection Level} name) This describes various permission
    flags such as \textit{normal}, \textit{development}, \textit{dangerous}, \textit{instant}, \textit{granted},
    \textit{revoked}, \textit{signature}, \textit{privileged}, etc.

    \item \textbf{Package Name.} Denotes the package name associated with the permission, i.e.\ the package that defined
    the permission.

    \item \textbf{Group.} The group name associated with the permission (if any). Several related permissions can often
    be grouped together.
\end{itemize}
%%!!>>

\subsection{Signatures Tab}\label{subsec:signatures-tab} %%##$signatures-tab-title>>
%%!!signatures-tab_1<<
\textbf{Signatures} are actually called signing information. An application is signed using one or more signing
certificates by the application developers before publishing it. The integrity of an application i.e.\ whether the
application is from the actual developer and isn't modified by other people can be checked using the signing information;
because when an application is modified by an unauthorised entity, the application cannot be signed using the original
certificates again because the signing information are kept private by the actual developer. Signing information can be
verified using checksums. Checksums are generated from the certificates themselves. If the developer supplies the
checksums for the signing certificates, they can be matched against the checksums generated in the \textbf{Signatures}
tab to verify the application. For example, if you have downloaded App Manager from GitHub or Telegram Channel, you can
verify whether the application is actually released by me by simply matching the following \textit{SHA256} checksum with
the one displayed in this tab:
%%!!>>
\begin{Verbatim}
320c0c0fe8cef873f2b554cb88c837f1512589dcced50c5b25c43c04596760ab
\end{Verbatim}

%%!!signatures-tab_2<<
Several hashing algorithms are used to generate checksums in this tab. They include \textit{MD5}, \textit{SHA1},
\textit{SHA256} and \textit{SHA512}.

\begin{danger}{Caution}
    Signing information should be verified using a reliable hashing algorithm such as \textit{SHA256}. DO NOT rely on
    \textit{MD5} or \textit{SHA1} checksums as they are known to generate the same checksums for multiple certificates.
\end{danger}
%%!!>>

\subsection{Shared Libraries Tab}\label{subsec:shared-libs-tab} %%##$shared-libs-tab-title>>
%%!!shared-libs-tab<<
\textbf{Shared libraries} tab lists all the legacy JAR dependencies as well as JNI (Java native interface) libraries.
For JNI libraries, it specifies platform (x86/x86\_64/ARM/AArch64), architecture (32/64 bit), object type (shared object
or executable), etc.
%%!!>>

\subsection{Other Tabs}\label{subsec:other-tabs} %%##$other-tabs-title>>
%%!!other-tabs<<
Other tabs list android manifest components such as features and configurations. A complete description
about these tabs will be available soon.
%%!!>>
