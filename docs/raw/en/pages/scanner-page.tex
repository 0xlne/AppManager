% SPDX-License-Identifier: GPL-3.0-or-later OR CC-BY-SA-4.0
\section{Scanner Page}\label{sec:scanner-page} %%##$section-title>>
%%!!intro<<
\textbf{Scanner page} appears after clicking on the \emph{scanner} button in the \hyperref[subsec:app-info-tab]{App Info
tab}. External APK files can also be opened for scanning from file managers, web browsers, etc.

It scans for trackers and libraries, and displays the number of trackers and libraries as a summary. It also displays
checksums of the APK file as well as the signing certificate(s).

\begin{danger}{Disclaimer}
    AM only scans an app statically without prejudice. The app may provide the options for opting out, or in some cases,
    certain features of the tracker may not be used at all by the app (e.g.\ F-Droid), or some apps may simply use them
    as placeholders to prevent the breaking of certain features (e.g.\ Fennec F-Droid). \textbf{The intention of the
    scanner is to give you an idea about what the APK might contain. It should be taken as an initial step for further
    investigations.}
\end{danger}

Clicking on the first item (i.e.\ number of classes) opens a new page containing a list of tracker classes for the app.
All classes can also be viewed by clicking on the \textit{Toggle Class Listing} menu. A sneak-peek of each class can be
viewed by simply clicking on any class item.

\begin{tip}{Notice}
    Due to various limitations, it is not possible to scan all the components of an APK file. This is especially true if
    an APK is highly obfuscated. The scanner also does not check strings (or website signatures).
\end{tip}

The second item lists the number of trackers along with their names. Clicking on the item displays a dialog containing
the name of trackers, matched signatures, and the number of classes against each signature. Some tracker names may have
$^2$ prefix which indicates that the trackers are in the \href{https://etip.exodus-privacy.eu.org}{ETIP} stand-by list
i.e.\ whether they are actual trackers is still being investigated.

The third item lists the number of libraries along with their names. The information are mostly taken from
\href{https://gitlab.com/IzzyOnDroid/repo}{IzzyOnDroid repo}.

\seealsoinline{\hyperref[subsec:tracker-classes-versus-tracker-components]{FAQ: Tracker classes vs tracker components}}
%%!!>>

\subsection{Missing Signatures}\label{subsec:missing-signatures} %%##$missing-signatures-title>>
%%!!missing-signatures<<
At the bottom of the page, there is a special item denoting the number of missing signatures (i.e.\ missing classes).
The missing signatures are the ones that AM has failed to match against any known libraries. The number itself has no
particular meaning as many libraries contain hundreds of classes, but clicking on the item will bring up a dialog
containing the signatures which is helpful in inspecting the missing signatures. \textbf{This feature is only intended
for people who know what a missing signature is and what to do with it, average users should just ignore it.}
%%!!>>
