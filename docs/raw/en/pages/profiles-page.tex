% SPDX-License-Identifier: GPL-3.0-or-later OR CC-BY-SA-4.0
\section{Profiles Page}\label{sec:profiles-page} %%##$section-title>>
%%!!intro<<
Profiles page can be accessed from the \hyperref[subsec:main-page-options-menu]{options-menu} in the main page.
It primarily displays a list of configured profiles along with the typical options to perform operations on them.
New profiles can also be added using the \textit{plus} button in the bottom-right corner as well as can be imported,
duplicated, or created from one of the presets. Clicking on any profile item opens its \hyperref[sec:profile-page]{profile page}.
%%!!>>

\subsection{Options Menu}\label{subsec:profiles-options-menu} %%##$options-menu-title>>
%%!!options-menu<<
There are two options menu in this page. The three dots menu in the top-right corner offers two options:
\textit{presets} and \textit{import}.
\begin{itemize}
    \item \textbf{Presets.} This option lists a number of built-in profiles that can be used as a starting point.
    The profiles are generated from the project \href{https://gitlab.com/W1nst0n/universal-android-debloater}{Universal
    Android Debloater}.\\
    \seealsoinline{\hyperref[subsec:faq:what-are-bloatware]{FAQ: What are bloatware and how to remove them?}}

    \item \textbf{Import.} This option can be used to import an existing profile that was previously exported from App Manager.
\end{itemize}

Long clicking on any profile item brings up another options-menu. It offers the following options:
\begin{itemize}
    \item \textbf{Apply now\dots.} This option can be used to apply the profile directly. When clicked, a dialog will be
    displayed where it is possible to select a \hyperref[subsubsec:profile-state]{profile state}.
    On selecting one of the options, the profile will be applied immediately.

    \item \textbf{Delete.} Clicking on delete will remove the profile immediately without any warning.

    \item \textbf{Duplicate.} This option can be used to duplicate the profile. When clicked, a dialog will be
    displayed where it is possible to set a name for the new profile. On clicking ``OK'', \hyperref[sec:profile-page]{profile page}
    will be loaded by duplicating all the configurations that this profile have. However, the profile will not be saved
    until it is saved manually.

    \item \textbf{Export.} Export the profile to an external storage. Profile exported this way can be imported via
    the \textit{import} option as mentioned above.

    \item \textbf{Create shortcut.} This option can be used to create a shortcut for the profile. When clicked, there
    will be two options: \textit{Simple} and \textit{Advanced}. When configured with the latter option, it prompts the
    user to select a profile state when the shortcut is invoked. The former option, on the other hand, always uses the
    default state that was configured when the profile was last saved.
\end{itemize}
%%!!>>
