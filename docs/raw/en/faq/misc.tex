% SPDX-License-Identifier: GPL-3.0-or-later OR CC-BY-SA-4.0
\section{Miscellanea}\label{sec:faq:miscellanea} %%##$section-title>>

\subsection{I don't use root/ADB. Am I completely safe from any harms?}\label{subsec:faq:no-root-no-harms} %%##$i-dont-use-root-adb-title>>
%%!!i-dont-use-root-adb<<
Yes. AM cannot modify any system settings without root or ADB over TCP\@.
%%!!>>

\subsection{How are the trackers and libraries are updated?}\label{subsec:faq:how-trackers-libs-updated} %%##$how-tracker-updated-title>>
%%!!how-tracker-updated<<
Trackers and libraries are updated manually before making a new release.
%%!!>>

\subsection{Are APKs deleted after installed?}\label{subsec:faq:apks-deleted-after-installed} %%##$apks-deleted-after-installed>>
%%!!apks-deleted-after-installed<<
No, APKs aren't deleted by App Manager after they are installed.
%%!!>>

\subsection{Any plans for Shizuku?}\label{subsec:faq:shizuku-support} %%##$shizuku-title>>
%%!!shizuku<<
App Manager's use of hidden API and privileged code execution is now much more complex and cannot be integrated with
other third party apps such as \href{https://shizuku.rikka.app}{Shizuku}. Here are some reasons for not considering
Shizuku (which now has Apache 2.0 license) for App Manager:
\begin{enumerate}
    \item Shizuku was initially non-free which led me to use a similar approach for App Manager to support both root
    and ADB
    \item App Manager already supports both ADB and root which in some cases is more capable than Shizuku
    \item Relying on a third-party app for the major functionalities is not a good design choice
    \item Integration of Shizuku will increase the complexity of App Manager.
\end{enumerate}
%%!!>>

\subsection{What are bloatware and how to remove them?}\label{subsec:faq:what-are-bloatware} %%##$bloatware-title>>
%%!!bloatware<<
Bloatware are the unnecessary apps supplied by the vendor or OEM and are usually system apps. These apps are often used
to track users and collect user data which they might sell for profits. System apps do not need to request any
permission in order to access device info, contacts and messaging data, and other usage info such as your phone usage
habits and everything you store on your shared storage(s).

The bloatware may also include Google apps (such as Google Play Services, Google Play Store, Gmail, Google, Messages,
Dialer, Contacts), Facebook apps (the Facebook app consists of four or five apps), Facebook Messenger, Instagram,
Twitter and many other apps which can also track users and/or collect user data without consent given that they all are
system apps. You can disable a few permissions from the Android settings but be aware that Android settings hides almost
every permission any security specialist would call potentially \emph{dangerous}.

If the bloatware were user apps, you could easily uninstall them either from Android settings or AM. Uninstalling system
apps is not possible without root permission. You can also uninstall system apps using ADB, but it may not work for all
apps. AM can uninstall system apps with root or ADB (the latter with certain limitations, of course), but these methods
cannot \emph{remove} the system apps completely as they are located in the \emph{system} partition which is a read-only
partition. If you have root, you can remount this partition to manually \emph{purge} these apps but this will break Over
the Air (OTA) updates since data in the system partition has been modified. There are two kind of updates, delta
(small-size, consisting of only the changes between two versions) and full updates. You can still apply full updates,
but the bloatware will be installed again, and consequently, you have to delete them all over again. Besides, not all
vendors provide full updates.

Another solution is to disable these apps either from Android settings (no-root) or AM, but certain services can still
run in the background as they can be started by other system apps using Inter-process Communication (IPC). One possible
solution is to disable all bloatware until the service has finally stopped (after a restart). However, due to heavy
modifications of the Android frameworks by the vendors, removing or disabling certain bloatware may cause the System UI
to crash or even cause bootloop, thus, (soft) bricking your device. You may search the web or consult the fellow users
to find out more about how to debloat your device.

From v2.5.19, AM has a new feature called \hyperref[sec:profile-page]{profiles}. The
\hyperref[sec:profiles-page]{profiles page} has an option to create new profiles from one of the presets. The presets
consist of debloating profiles which can be used as a starting point to monitor, disable, and remove the bloatware from
a proprietary Android operating system.

\begin{warning}{Note}
    In most cases, you cannot completely debloat your device. Therefore, it is recommended that you use a custom ROM
    free from bloatware such as Graphene OS, Lineage OS or their derivatives.
\end{warning}
%%!!>>
